\documentclass[a4paper]{article}
%\documentclass[notimes,pnumplain]{abnt} % Pacote ABNT, que deixar� tudo de acordo com a ABNT

\pagestyle{plain}
\renewcommand{\baselinestretch}{1.5}

\usepackage[latin1]{inputenc}
\usepackage[T1]{fontenc}
\usepackage[brazil]{babel}
\usepackage[alf]{abntcite} % Carregamos o pacote abntcite com a op��o alf, ou seja, cita��es alfanum�ricas
\usepackage{amsmath,amstext,amssymb} %Para usarmos o modo matem�tico e contruir as matrizes e f�rmulas


\title{Projeto}
\author{Michel Marques Farah}

\begin{document}
\maketitle
\newpage
\tableofcontents
%\listoftables
%\listoffigures
\newpage
\include{introducao}
\section{Material e M�todos}
\subsection{Dados}
Os dados utilizados ser�o provenientes de fazendas que integram o programa de melhoramento gen�tico da Conex�o Delta G, contando com 2000 animais machos da ra�a Nelore terminados em confinamento com idade pr�xima a dois anos, filhos de cerca de 100 touros, provenientes de seis rebanhos, todos com informa��es de desempenho e genealogia dispon�veis. Destes, 980 animais ser�o genotipados utilizando um painel de aproximadamente 800000 SNP do BovineSNP BeadChip (High-Density Bovine BeadChip) \cite{illumina}.

Na an�lise de controle de qualidade, os SNPs ser�o submetidos a um teste de probabilidade, assumindo uma distribui��o de 1ui-quadrado e n�vel de signific�ncia de P<0,05.

Os SNPs com desvios significativos do equil�brio de Hardy-Weinberg ser�o exclu�dos da an�lise, assim como SNPs com baixa frequ�ncia ({\it minor frequency alleles}: MAF) e aqueles que estejam genotipados em menos de 50\% da popula��o. Todas as an�lises ser�o realizadas utilizando o conjunto de dados completo e os registros ser�o analizados utilizando um modelo animal.

\subsection{Matriz de parentesco pedigree-gen�mica}

A matriz de parentesco pedigree-gen�mica (H) ser� utilizada para substituir a matriz de parentesco (A). A matrix H ser� obtida pela combina��o das informa��es de pedrigree e gen�mica \cite{AGUILAR2010}

\begin{equation}
H^{-1}=A^{-1}+
\begin{bmatrix}
0 & 0 \\
0 & G^{-1}-A^{-1}_{22}
\end{bmatrix},
\end{equation}

onde $G^{-1}$ � a inversa da matriz de parentesco gen�mica e $A^{-1}_{22}$ � a inversa da matriz de parentesco dos animais genotipados. As compara��es ir�o envolver v�rias matrizes de parentesco gen�mica (G), de modo geral, para a obten��o da matriz G foi usada a equa��o descrita por \citeonline {VANRADEN2008} :

\begin{equation}
G = \frac{(M-P)(M-P)'}{2 \sum_{j=1}^{m} p_{j} (1-p_{j})},
\end{equation}

em que $M$ � uma matriz al�lica com $m$ colunas (m=n�mero total de marcadores) e $n$ linhas (n=n�mero total de animais genotipados); e $P$ � uma matriz que cont�m a frequ�ncia do segundo alelo ($p_{j}$), expresso como $2p_{j}$.

A matriz $M$ ser� preenchida com -1 se o gen�tipo do animal $i$ para o SNP $j$ for homozigoto, pode ser preenchido com 0 se for heterozigoto ou preenchido com 1 se o gen�tipo for o outro homozigoto. As frequ�ncias al�licas utilizadas ser�o: 0,5 para todos os marcadores (G05), a frequ�ncia m�dia do menor alelo (GMF) e a frequ�ncia observada do alelo de cada SNP (GOF).

\subsection{Matriz IBS}

Utilizando o m�todo descrito anteriormente, para obter a matriz $M$, de acordo com \citeonline{VANRADEN2008}, os elementos da diagonal de $MM'$ representam o n�mero de locus em homozigose para cada indiv�duo e os elementos fora da diagonal representam o n�mero de alelos compartilhados por parentes. Em contraste, os elementos da diagonal de $M'M$ demonstram o n�mero de indiv�duos homozigotos para cada locus e fora da diagonal indicam o n�mero de vezes que os alelos em diferentes loci foram herdados pelo mesmo indiv�duo.

\subsection{Matriz IBD}

Devido a complexidade para a obten��o da matriz gen�mica com informa��es de alelos id�nticos por descend�ncia, para a an�lise desta matriz ser� utilizada a simula��o de dados.

\include{cronograma}
\bibliography{projeto}
\end{document}
